% COMMANDS
\newcommand{\reporttitle}{CDS Lab Report - Himeno benchmark}
\newcommand{\hb}{\textit{Himeno Benchmark}}
\newcommand{\jm}{\textit{Jacobi method}}

% Use only LaTeX2e, calling the article.cls class and 12-point type.

\documentclass[12pt]{article}


% Use times if you have the font installed; otherwise, comment out the
% following line.

\usepackage{times}

% For set of numbers in math
\usepackage{amsfonts}

% For listings (code blocks)
\usepackage{listings}
\lstset
{ %Formatting for code in appendix
    basicstyle=\footnotesize,
    numbers=left,
    stepnumber=1,
    showstringspaces=false,
    tabsize=4,
    breaklines=true,
    breakatwhitespace=false,
}

% For references
\usepackage{nameref}
\usepackage{hyperref}


% The following parameters seem to provide a reasonable page setup.
\usepackage{titlesec}

\topmargin 0.0cm
\oddsidemargin 0.2cm
\textwidth 16cm 
\textheight 21cm
\footskip 1.0cm

\titlespacing*{\section}
{0pt}{6.5ex plus 1ex minus .2ex}{1.3ex plus .2ex}
\titlespacing*{\subsection}
{0pt}{6.5ex plus 1ex minus .2ex}{1.3ex plus .2ex}
\titlespacing*{\subsubsection}
{0pt}{6.5ex plus 1ex minus .2ex}{1.3ex plus .2ex}

%The next command sets up an environment for the abstract to your paper.

\newenvironment{sciabstract}{%
\begin{quote} \bf}
{\end{quote}}


% If your reference list includes text notes as well as references,
% include the following line; otherwise, comment it out.

\renewcommand\refname{References and Notes}

% The following lines set up an environment for the last note in the
% reference list, which commonly includes acknowledgments of funding,
% help, etc.  It's intended for users of BibTeX or the {thebibliography}
% environment.  Users who are hand-coding their references at the end
% using a list environment such as {enumerate} can simply add another
% item at the end, and it will be numbered automatically.

\newcounter{lastnote}
\newenvironment{scilastnote}{%
\setcounter{lastnote}{\value{enumiv}}%
\addtocounter{lastnote}{+1}%
\begin{list}%
{\arabic{lastnote}.}
{\setlength{\leftmargin}{.22in}}
{\setlength{\labelsep}{.5em}}}
{\end{list}}


% Include your paper's title here

\title{\reporttitle} 


% Place the author information here.  Please hand-code the contact
% information and notecalls; do *not* use \footnote commands.  Let the
% author contact information appear immediately below the author names
% as shown.  We would also prefer that you don't change the type-size
% settings shown here.

\author
{Fritz Louis Wilke\\
\\
\normalsize{\textbf{Faculty:} Computer Science} \\
\normalsize{\textbf{Study Program:} M. Sc. Computer Science} \\
\normalsize{\textbf{E-Mail:} fritz.wilke@tu-dresden.de} \\
\normalsize{\textbf{Mat-Nr.:} 4536116}
}

% Include the date command, but leave its argument blank.

\date{}



%%%%%%%%%%%%%%%%% END OF PREAMBLE %%%%%%%%%%%%%%%%
\begin{document}
\baselineskip16pt
\maketitle
\thispagestyle{empty}

% The report content starts here

\clearpage
\setcounter{page}{1}

% INTRODUCTION
\section{Introduction}
\label{sec:introduction}

\subsection{The Problem}
\label{ssec:problem}

The \hb{} is - as the name implies - a benchmarking program. It was developed by Dr. Ryutaro Himeno at the RIKEN Institute in 1996, but the \href{https://i.riken.jp/en/supercom/documents/himenobmt/}{source code} is still available today. Its goal is to provide a comparable, overall performance benchmark between different machines and architectures. This includes the memory throughout, caching behaviour and CPU performance (FLOPS).

The program requires four inputs. The $rows$, $cols$ and $deps$ define the size of the 3D-matrices to work with and $nn$ is used to specify the number of \textit{Jacobi} iterations performed. The \jm{} is an iterative algorithm to determine the solutions of a system of linear equations. Usually, the \jm{} is executed until the solution converges, but in this program the number of iterations is defined by $nn$.

The output of the program is calculated in the last iteration of the \jm{} and is called the \textit{Gosa number}. In a nutshell, the \textit{Gosa number} is the sum of the squared average of the direct neighbours of each voxel $v_{ijk} \in M$, where $M$ is the current three dimensional working matrix.

The task of this lab is to find a way to parallelize the described program in a way that it scales well with an increasing number of available CPUs. To achieve this, strategies of concurrent and distributed systems programming are used to plan, implement and evaluate the parallelized program. The provided output must match the one given by the sequential solution, though the algorithm itself may be modified. Note that it is not the goal of the parallelized program to provide a comparable benchmark, as it was of the original program.


\subsection{The Code Base}
\label{ssec:code-base}


\end{document}




















